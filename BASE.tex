\documentclass[12pt, twocolumn, a4paper]{article}
\usepackage{multicol, lipsum}
\usepackage[utf8]{inputenc}
\usepackage{cite}
\usepackage{amsmath}
\usepackage{amsfonts}
\usepackage{amssymb}
\usepackage{graphicx}
\usepackage[a4paper, left=3cm, right=2cm, top=3cm, bottom=2cm,
		headsep=1cm, footskip=2cm]{geometry}

\begin{document}
	\title{SystemVerilog Vocabulary Extractor}
	\author{Filipe C. Cavalcanti\\ Leandro de S. Albuquerque\\
	Orientador: Tio Kat}
	\maketitle
	
	\section{ABSTRACT}
	
	\section{RESUMO}
	\quad texto corrido\cite{Antoniol2007}
	outra citação\cite{Alfke1943}
	asdfsagfa sdfsadf sadfasdf asdfasfsg
	safdasdg tasdfsa dafsasadf slsss ssssss ssssss sssssssd
	sdffffff ffffff ffffff ffffffff ffffff fffffffff fffsfd
	sdffff ffffffffffff ffff ffffff fffffff fffffff ffffffff fffffff ffffff ffff ffff
	sdfasf asdfasdfff ffffffffffff ff fffffff ffffff ffff fff fffff
	asfd ddddd dddddddd dddddddddddd ddddddd ddddddddd dddddddd dddddddd dddddd
	dsaa ass ssssss sss ssssss ssss ssssssss sssss sssssss ssss sssssss sssss ssss ssss
	assssss ssssssssss sssssssss ssssss ssssssss ssssssssssss sssssssss sssssss
	asssss ssssss ssssssss sssssssss ssssssssss sssssssssss sssssssdddddddddddd
	adssssss sssssss ssssss ssssss sssssssf ffffff fffff ffff fffffd
	
	
	\section{Introdução}
\quad Quando Verilog foi criado em meados de 1980, o tamanho típicos dos designs (circuitos lógicos) era na ordem de 5 a 10 mil portas lógicas, o método de design dos circuitos era usando esquema gráfico, e a simulação estava começando a ser uma ferramenta essencial para verificação\cite{sutherland2006}. A linguagem Verilog continuou a evoluir com a tecnologia de design e verificação ate que, em 2002 surge SystemVerilog, sendo esta linguagem uma significante melhoria de Verilog\cite{sutherland2006}. A partir disto, Como a complexidade de sistemas digitais modernos continua a aumentar exponencialmente, tem-se que as metodologias de design RTL estão crescendo também\cite{Marc-Andre} e \cite{Hahanov2008}.

Com tal avanço, elevou-se o nível de abstração no desenvolvimento de hardware através de uma linguagem de descrição e verificação de hardware (HDVL), de tal forma que, fez-se necessário o uso de ferramentas de análise de informações que antes eram somente do escopo da engenharia de software.

Nos últimos anos, SystemVerilog e SystemC estão sendo extensivamente usadas para design e verificação na industria VLSI(Very Large Scale Integration)\cite{Kumar2014}. Sendo nosso foco SystemVerilog,que é uma unificação entre, design de hardware, e linguagem de verificação\cite{IEEEComputerSociety2013}, SystemVerilog permite o uso de uma linguagem unificada para especificações abstratas e detalhadas do design e verificação, também em \cite{IEEEComputerSociety2013}.

Umas das principais fontes de informações em um código fonte é o vocabulário do mesmo. O vocabulário também denominado de léxico do código em \cite{Host2007} e \cite{Antoniol2007}, consiste no conjunto de termos repetidos ou únicos que compõem identificadores e que estão presentes no textos dos comentários\cite{Abebe2009}.

Usando os princípios da engenharia reversa como uma coleção de metodologias e técnicas capazes de realizar a extração e abstração de informações\cite{BENEDUSI1992225}, propõe-se neste trabalho uma ferramenta que possibilita a extração de vocabulário para SystemVerilog, além de fundamentar o termo \textit{Vocabulário de Hardware}.

	\section{Background}
\quad Graças aos atuais designs eletrônicos baseado em HDL, metodologias e ferramentas para simulação, síntese, verificação, modelagem física e teste pós-fabricação agora estão bem inseridos e são essenciais para designers digitais \cite{Navabi2015}. Nos últimos anos as linguagens de descrição e verificação de hardware tornaram-se tão importantes para a modelagem de sistemas digitais, quanto as linguagens de programação o são para a engenharia de software. 
	\subsection{O Que é Uma HDVL?}
\quad HDVL (Hardware Description and Verification Language), podemos abstrair como um único ambiente para design e verificação de sistemas digitais, em \cite{Flake} uma HDVL representa hardware digital em vários níveis de abstração.
	\subsection{O Hardware Como Um Software}
\quad 
	\subsection{Vocabulário de software}
\quad
	\section{SystemVerilog Vocabulary Extractor}
	\section{Resultados e Discussões}
	\bibliography{refs}
	\bibliographystyle{abbrv}
	
\end{document}