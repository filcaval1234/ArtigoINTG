\documentclass[12pt, twocolumn, a4paper]{article}
\usepackage{multicol, lipsum}
\usepackage[utf8]{inputenc}
\usepackage{cite}
\usepackage{amsmath}
\usepackage{amsfonts}
\usepackage{amssymb}
\usepackage{graphicx}
\usepackage[a4paper, left=3cm, right=2cm, top=3cm, bottom=2cm,
		headsep=1cm, footskip=2cm]{geometry}

\begin{document}
	\title{SystemVerilog Vocabulary Extractor}
	\author{Filipe C. Cavalcanti\\ Leandro de S. Albuquerque\\
	Orientador: Tio Kat}
	\maketitle
	
	\section{ABSTRACT}
	
	\section{RESUMO}
	texto corrido\cite{Antoniol2007}
	outra citação\cite{Alfke1943}
	\section{Introdução}
\quad O primeiro FPGA(Field Programmable Gate Array) comercialmente disponível  foi lançado em 1985, fornecendo 64 blocos de lógica configurável e 58 blocos de entrada e saída de seus 85.000 transistores\cite{Alfke1943}. Nos tempos modernos, os FPGAs se tornaram chips com bilhões de transistores, fornecendo milhares de bits de memória on-chip, dezenas de milhares de registradores e centenas de blocos DSP(Digital Signal Processor)\cite{Marc-Andre}. A partir disto, Como a complexidade de sistemas digitais modernos continua a aumentar exponencialmente, tem-se que as metodologias de design RTL estão crescendo também\cite{Marc-Andre} e \cite{Hahanov2008}.

Com tal avanço, elevou-se o nível de abstração no desenvolvimento de hardware através de uma linguagem de descrição e verificação de hardware (HDVL), de tal forma que, fez-se necessário o uso de ferramentas de análise de informações que antes eram somente do escopo da engenharia de software.

Nos últimos anos, SystemVerilog e SystemC estão sendo extensivamente usadas para design e verificação na industria VLSI(Very Large Scale Integration)\cite{Kumar2014}. Sendo nosso foco SystemVerilog,que é uma unificação entre, design de hardware, e linguagem de verificação\cite{IEEEComputerSociety2013}, SystemVerilog permite o uso de uma linguagem unificada para especificações abstratas e detalhadas do design e verificação, também em \cite{IEEEComputerSociety2013}.

Umas das principais fontes de informações em um código fonte é o vocabulário do mesmo. O vocabulário também denominado de léxico do código em \cite{Host2007} e \cite{Antoniol2007}, consiste no conjunto de termos repetidos ou únicos que compõem identificadores e que estão presentes no textos dos comentários\cite{Abebe2009}.

Usando os princípios da engenharia reversa como uma coleção de metodologias e técnicas capazes de realizar a extração e abstração de informações\cite{BENEDUSI1992225}, este trabalho propõe uma ferramenta que possibilita a extração de vocabulário para Systemverilog, além de fundamentar o termo \textit{"Vocabulário de Hardware"}.

	\section{Background}
	\subsection{O Que é Uma HDVL?}
	\subsection{O Hardware Como Um Software}
	\subsection{Vocabulário de software}
	\section{SystemVerilog Vocabulary Extractor}
	\section{Resultados e Discussões}
	\bibliography{refs}
	\bibliographystyle{acm}
\end{document}